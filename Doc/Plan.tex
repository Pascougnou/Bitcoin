\documentclass{acm_proc_article-sp}

\begin{document}

\title{BitCoin a Peer-to-Peer payment solution}
\subtitle{[Security Considerations]
\titlenote{This paper is the conclusion of a second year project in
a French engeniering school Ensimag for more informations see
\texttt{www.ensimag.grenoble-inp.fr/} or 
\texttt{www.ensiwiki.ensimag.fr/index.php/4MMPCRYPTO\_2013}}}

\numberofauthors{3}
\author{
\alignauthor
Jean\-Guillaume Dumas\\
       \affaddr{University Joseph Fourier Grenoble}\\
       \email{Jean-Guillaume.Dumas@imag.fr}
\alignauthor
Pascal Sygnet\\
       \affaddr{Grenoble INP Ensimag}\\
       \email{Pascal@Sygnet.info}
\alignauthor 
Vincent Xuereb\\
       \affaddr{The Th{\o}rv{\"a}ld Group}\\
       \email{Vincent.Xuereb@Phelma.Grenoble-inp.fr}
}


\date{16 June 2013}
\maketitle

\section{Project}
In this project we propose to investigate a novative system of payment
the BitCoin. After compiling some documentation we decided to investigate
three (3) different security problems:
\begin{itemize}
\item[-] the first one is the heart of this system : forgery of a proof of work;
\item[-] the second is the notion of \textbf{anonimity} 
during a transaction;
\item[-] the third issue we choosed to focus on was an anomaly in
the transaction record detected by Dorit Ron \& Adi Shamir in their 2012 paper
\footnote{\texttt{www.eprint.iacr.org/2012/584.pdf}}.
\end{itemize}
\section{Introduction}
First draft of the article with simple idea and short descriptions.

\section{A Peer-to-Peer payement System: BitCoin}
\subsection{Traditionnal Banking}
A traditionnal electronic cash system (via internet or any kind of network)
is based on a central authority \textbf{the mint}. 
This mint (or bank) is aware of all transactions, of the balance of each 
and evry account in his own network and is responsible of security and anonimity 
of the transactions. To ensure privacy the bank keeps informations only bettween
the involved parties. The main advantage is the simplicity of protecting a 
transaction as the only requiered informations to spend money is a single 
identification as the bank has access to both balance of accounts and the time of 
the transaction. The entire intelligence (verification and issuing the keys) is
tranfered into the mint so the users only need to know their own key (\& id).
A real problem of this system is that it relies on a single central mint. If it
was to collapse evry history of transactions and evry amount of money would be 
lost without any chance of recovery.
That's why people started to envestigate different solutions like BitCoin (based on a peer-to-peer network) for example. Here is a summary of the specifications of some "<famous"> electronnic currencies. 

\begin{table}[h]
\centering
\caption{Some Electronic Currencies specifications}
\begin{tabular}{|p{1,5cm}|p{2cm}|p{1,5cm}|p{1,5cm}|p{1cm}|} \hline
&Mint &Public Transactions &Anonimity &PtP\\ \hline
Bank & Yes & No & Yes & No\\ \hline
Ripple & No & Yes & depends & Yes\\ \hline
KARMA & distributed & No & Yes & Yes\\ \hline 
PPay & A user = Mint for evry coin he generates & No & Yes & Yes\\ \hline
BitCoin & No & Yes & we'll see & Yes\\ \hline
\end{tabular}
\end{table}

In the rest of this article we'll focus on one particular currency, the BitCoin.

\subsection{BitCoin specificities}
!!Proof of work!!
An attacker with more CPU power than all the other nodes on the network
could benefit from it by mining (costs of maintaining such CPU
power on electricity connection??) said Nakamoto. Simple verification 
trust a small number of nodes (security flow?)
BitCoin is a peer-to-peer electronic currency system first described
by S. Nakamoto in 2008 \footnote{www.bitcoin.org/bitcoin.pdf}. It's based on
digital signature to prove ownership and an history of transactions publicly 
available to avoid double-spending. This history is shared using a peer-to-peer 
network and users agree on it using a proof-of-work system.
A BitCoin (or simply coin) is a chain of digital signatures, each owner signs
a hash of the prevous transaction and the public key of the next owner and 
adds this at the end of the coin. To avoid double-spending the system implements
a distributed timestamp server based on a proof-of-work system. Each time a 
node is notified of transactions, it puts them in a block and then hashes a
nonce (using SHA-256) and the previous hash in the proof-of-work incrementing 
it until it start with a predetermine number of zeros.     


\section{Forging a proof of work}
\subsection{Theoric complexity}
Present the theory developped in "Bitcoin: A Peer-to-Peer Electronic Cash System
by Satoshi Nakamoto and deduce the complexity of an attack, based on 
redoing a full proof of work.

\subsection{Effective complexity}
Implement a begenning of redoing a proof of work, then monitor the 
beginning (the entire history of transactions is available online)
and extrapolate to deduce an effective complexity.
\subsection{An other way to forge a transation}
The bitcoin system of validation of a transaction allows a user to ask others if the transation is valid. To attack you may not need to redo the proof of work but just give false information. issue we can address:
\begin{itemize}

\item[-] who utilise the peer validation;
\item [-]how much node is it necessary to have to make this methode of validation unreliable;
\item[-] how to detect such attak.

\end{itemize}

\section{Use attack to destroy anonimity or steal identities}
Present (and analyse) the complexity of the attack described in
the Ron \& Shamir paper. Parse the history of transactions so we can work whith it. Modify it to gain informations (merging public key belonging to the same user). Then see how to gather external information on bitcoin user. Question to answer : is it possible (and by who?) to discover the identity of bitcoin users.
\subsection{The case of multi-signature transaction}
Standart bitcoin transaction are single-signature transaction, however the Bitcoin network allows multi-signatures transaction (M-ofN transaction). We can espect ,in these transaction, multiple input from multiple person. We can adapt the parser to know what is the ratio multiple/single signature tansaction.
\subsection{Other issues}
several user can provide input and sign the same output(the transaction have now multiple input made by different entity.
Sevices as Mtgox might use this feature.
\section{Anomaly in the BitCoin transaction flow}
 Using the publicly available history of transaction we can analyse anomaly in the transation flow. In this section we can utilise what we will learn about anonymity and security of bitcoin user to better understand these anomaly. 
\section{links and refenreces}

\balancecolumns
% That's all folks!
\end{document}
